%本篇論文探討如何利用單張彩色影像來重建出三維人臉模型。我們的方法是利用彩色影像來取代灰階值去建立張量模型(tensor model),在人臉資料庫中,是用典型相關分析(Canonical correlation analysis)是來建立彩色影像與深度資訊的對應關係,一旦建立好屬於各自的對應關係後,在重建人臉的過程中,就只需要單張彩色影像即可利用典型相關分析的對應關係來推算出正確的深度資訊。實驗中,我們的方法可以在不同的光線環境跟人臉角度得到不錯的效果。

本篇論文利用深度攝影機重建點雲環境,取得環境資訊作定位。研究方法主要分成三個階段:建置點雲環境、虛擬照片拍攝及資料庫建置、特徵點定位,建置點雲環境的階段主要利用RANSAC作初步的點雲重合,在
利用 Graph Slam 作拍攝路徑的最佳化,使得深度重合能夠擺放在理想的位置。藉由虛擬照片建置的影像資料庫,能夠取得一般相機所缺乏的定位資訊,像是距離景物較遠的距離,抑或是被障礙物遮蔽所導致特徵
資訊缺乏的情形發生。在最後特徵點定位的階段,利用SIFT將虛擬相片與代訂為相片作比對,求出的特徵點根據位置及其夾角進行三角定位。在實驗中,我們方法改善了定位成功的覆蓋率,以及平均誤差,說明照片
的拍攝位置與角度,都會影響定位資訊的取得,藉由改善相機分布與角度的情況,能夠有效的減低定位誤差。
