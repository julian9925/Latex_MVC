\section{研究背景介紹}

	機器人相關領域研究在近年來相關廣泛,有許多的研究應用在生活上面,像是定位、導航、視覺感測等,在定位的研究上像是現在的GPS定位,在只要
有衛星訊號的環境上就可以運用定位系統,在陌生的環境下可以藉由定位了解到自身所在位置相關的資訊。定位資訊的輔助下,藉由路徑規劃,即可完成運送
、傳遞的任務,像是車用導航,輔助駕駛到達目的地。室外定位除了利用GPS外,也有利用紅外線及雷達等系統,但GPS最為廣泛,因為佈及全世界的視野,不須
其他儀器輔助。其他像是紅外線及雷達等的儀器,遍布沒有這麼廣泛,較無法運用在日常生活中,較常用在私人用途,像是飛機、軍艦等的導航系統。

	室內定位的技術多用於訊號接收定位,需要設備上的支援,像是紅外線、聲納\cite{Bekkerman2006}、藍牙、Wi-Fi\cite{Chin-HengLim2007}
以及RFID\cite{Zhou2009}等室內定位的技術,這些技術的共同缺點都是訊號容易受到環境的干擾,使得無法精確地定出定位位置,誤差範圍過大(5\~10m),
為了替代訊號在定位上的使用,之後室內定位的研究轉往在影像上的特徵當作定位來源。影像定位藉由影像上的特徵作為定位處理的依據,影像特徵的擷取分為兩
大部分,一為區域導向(Region based),一為特徵點導向(feature based)\cite{Zitová03}。區域導向代表在兩張相對圖片中,找出相對應畫素的關係性,
像是以梯度向量(Gradient)強度,或是色彩像素上的強度作為特徵描述。另外一種特徵點導向是以lowe\cite{Lowe2004}所提出尺度不變特徵轉換(SIFT)為前
提,所使用擷取特徵的方法。

%影像特徵描述
	SIFT特徵點須具備以下幾個條件:特徵點的識別度高,具有獨特性、容易匹配、對各種的破壞具有不變性。基於這上述三項理由,使得SIFT比一般區域導向的特徵
更常用在定位演算法中。另外也有其他特徵點擷取的方法,像是Harris角點偵測法[3],不過其最大的缺點就是本身並非尺度不變,因為本身尺度上的限制,使得Harris
角點偵測法鮮少出現在影像定位文獻中。SIFT除了在特徵點擷取上效果優良外,在計算特徵點位置周圍區塊像素的影像梯度(gradient)後,求出方向性並與周圍區塊像
素的方向統計過後濃縮成特徵點向量,使得SIFT比其他區域導向的特徵描述,在對抗破壞與干擾上有更好的效果。

%影像定位的方法及應用
	影像定位除了在特徵點的擷取方法的不同上,定位方式也有所差異,像是利用機率方式Monte Carlo Localization(MCL)解決定位的問題\cite{Thrun2001},
或者是傳統的三角定位(Triangulation)\cite{Cobzas2003},都利用特徵描述的不同特性,完成定位。透過影像定位,除了讓機器人了解自己的位置外,也可以運用在
建置環境地圖上,像是SLAM(Simultaneously Localization And Mapping)關於定位與建置地圖一系列的問題方向,都在影像定位與特徵點擷取的相關技術做研究,
使得影像定位在不同的領域上有更多的應用發展。
	
%機器視覺與點雲上的應用
	3D環境模型建置有許多種的方法,在傳統的方法像是利用雷射掃描深度,利用渲染(Rendering)的技術,透過材質(texture)拼貼(mapping)在立體維度中,達成
立體視覺的效果。現在利用深度攝影機(Kinect)將深度記錄下來,與照片的像素作對應,使得每張照片的像素都有三維座標與其對應,這種三維座標的紀錄方式稱為點雲
(Point Cloud)。現在許多研究透過點雲應用在立體視覺上面,在\cite{Rusu2011}中,將點雲與機器人作結合,將膽雲應用在建置環境地圖或是定位、導航等用途。
這種3D環境建置的方法可以節省在Rendering所花費的計算時間,在者儀器的取得也比雷射或其他IR機器來的便利,更可將這些技術應用在日常生活上。
	
\section{研究動機}

	室內定位訊號受到環境上的阻礙,在比較複雜的室內環境下,會因為訊號不穩定而導致定位精準度下降,又受限於設備上的限制,無法落實在日常生活上。
我們想避免運用到需要訊號定位的設備,利用日常可隨身攜帶的相機當作我們定位的資訊,藉由影像特徵的幫助下完成室內定位。傳統的訊號定位,如:Wifi、RFID、等,
都需要透過設備上存取訊號資料,但在一些特定區域可能環境不能夠放置設備,抑或是會受到其他訊號機器干擾,在室內環境下,我們把研究方向改為影像定位的方面做改進。

	影像定位雖有比訊號定位較高的精準度,但是受限於拍照位置與角度上的限制,不能完整的涵蓋室內環境的資訊,在我們的研究想還原定位的3D環境,利用近年來常用的
Kinect深度感測攝影機幫助我們建置3D環境,一來可以知道室內環境的全貌,其次也可以藉由環境資訊了解自身所在的位置,增進影像定位的利用與便利性。

\section{研究目的}
		
	為使得影像定位能有更準確且更高覆蓋率的的定位,我們結合了3D環境的技術幫助我們達成目的,在重建的定位環境下設置虛擬相機取得照片,藉由均勻分布
在定位環境下,取得比相機拍攝更好的影像來源,除了均勻的分布位置外,我們根據虛擬相機所在的位置作深度檢測,當距離障礙物過近的情況下,能夠自動的旋轉
相機角度拍攝更好的虛擬影像。我們研究主對於室內影像定位要改善三個分針:
\begin{enumerate}
	  \item 均勻的相機分布取得充分的影像資料
	  \item 增加定位環境內可定位的覆蓋範圍
	  \item 改善整體定位的平均誤差
\end{enumerate}
	藉由改進這三個方針,使得影像資料能被更有效率的運用,改善因為隨機分布所導致影像資料的不完全,使得影像定位不會受到環境的阻隔而導致定位誤差過大,或是沒有相關資料
導致無法定位的情形發生。

\section{論文架構}

	在本篇研究中,第二章針對現在的影像定位技術相關研究與SLAM問題做探討,藉由之前影像定位研究的方法提出改進的目標,也分析解決SLAM問題的方法怎麼應用在建置
3D環境上,藉由之前的研究參考與改進,提出我們研究的方法	。第三章進入我們研究的核心,描述我們如何建置3D虛擬環境,並且說明虛擬照片取出的來源及過程,以及最後怎麼
運用到三角定位上。第四章針對我們的研究與SIFT影像定位做比對,在針對固定放置物品的定位環境下,離物品由近而遠比較定位成果,說明在我們方法改善了那些目標,並針對
一般室內環境之下,分別在客廳、廚房、與一般房間做室內定位比較,比較定位的精準度,以及定位在誤差下成功的覆蓋率。在最後一章,針對我們的貢獻以及未來能夠改進的目標
做說明。


