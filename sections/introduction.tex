%\section{Introduction}
%人臉重建會遇到的問題及方法種類
3D face reconstruction from 2D images is a challenging problem in computer vision.
%, because of inaccuracy, time-varying, time-consuming, unknown surface information. 
Over the years, several 3D face reconstructing approaches from 2D images have been proposed. Methods used to acquire depth information of faces including stereo vision\cite{seitz2006comparison}, structure from motion \cite{tomasi1992shape}, 3D morphable model based methods \cite{romdhani2003efficient}, shape from shading techniques \cite{zhang1999shape}, and tensor \cite{vasilescu2002multilinear, kolda2009tensor}.


% stereo vision 方法特性
In methods using stereo vision, selecting proper corresponding points is crucial because inaccurate corresponding points greatly affect the accuracy of depth estimation.  Corresponding points can be used to measure \textit{disparity}, which is the difference between the corresponding points of left and right images. Disparity is then used to evaluate depth, which is the distance to the target object.
%Okutomi														
Okutomi \cite{okutomi1993multiple} proposed a method to match corresponding points by computing the sum of squared-difference (SSD) with different baselines by a lateral displacement of a camera.
%Hansen
Recently, Hansen \cite{hansen20103d} developed a new 3D face capture algorithm using Photometric Stereo (PS) device with visible light or near infrared light (NIR). In general, the NIR light sources offer better performance than the visible light.

%structure from motion 方法特性
In methods using structure from motion, the three-dimensional structure of an object is found by analyzing local motion signals over time. It can recover 3D structure from the projected 2D motion field of a moving object.
%Torresani
Torresani et al.\cite{torresani2008nonrigid} proposed a reconstruction method using a Probabilistic Principal Components Analysis (PPCA) shape model and an estimation algorithm that estimates 3D shape and motion for each time moment simultaneously. 
%3D morphable model 
In 3D morphable model (3DMM) based methods, by given a large database of 3D face models, any arbitrary face can be generated by morphing in the database.		
%Romdhani
Romdhani \cite{romdhani2003efficient} proposed a 3D alignment algorithm to recover the facial shape and texture parameters from the appearance features. However, the 3D face alignment requires manual initialization and the reconstructing speed was not yet suitable for real time face recognition systems. 					
%Hu
To speed up the fitting algorithm, Hu \cite{hu2004automatic} proposed a fully automatic linear algorithm to recover the shape information according to sparsely corresponded 2D facial feature points, but the texture geometry extracted from a single image could not cover the whole face.
%Le 
Le et al.\cite{le2010accurate} proposed an algorithm which reconstructs the 3D shape and texture of human faces from stereo images captured from calibrated cameras. Using PCA-based morphable model and iterative optimization process, their algorithm fits arbitrary views of faces to a dense 3D model from a sparse 3D point set.						
%shape from shading
Shape from shading (SFS) methods use shading information to reconstruct the three-dimensional object. It needs to solve unknown lighting source and albedo (reflection coefficient) from object surface.
%Kemelmacher
Kemelmacher \cite{kemelmacher20113d} proposed a novel method for recovering 3D faces from a single image and single 3D reference model of a different person’s face.

%介紹tensor
In recent years, several tensor-based methods are proposed for face recognition and reconstruction \cite{vasilescu2002multilinear, kolda2009tensor}. 
%介紹Lei的tensor model及使用NIR image的優點
Lei et al. \cite{lei2008face} recovered faces from a single image. \cite{lei2008face} uses a single near infrared (NIR) image as input and constructs a mapping from the NIR tensor space to 3D tensor space.

%說明動機
In the NIR imaging system environment, \cite{lei2008face} is able to obtain good performance. 
However, because of the fact that cost constrain and the NIR imaging system is not easy to obtain, the grayscale images are substituted for NIR images. 
When we use the grayscale images as input, different lighting sources and other factors due to cameras affect the computed result. 
The grayscale image is an image the value of each pixel carries only intensity information. 
When lighting sources changes strongly, grayscale images cause some noisy and redundant information, because of grayscale images are sensitive to environmental lighting variation. 
It can cause negative effect when mapping between image and depth information.

%XXX Although \cite{lei2008face} is able to obtain good performance in the NIR imaging system environment, when %intensity images are substituted for 
%    NIR image, lighting sources and other factors due to cameras degrade the computed result. 
%In uncontrolled environment, a person's intensity image may appear differently because of intensity image are %sensitive to environment lighting variation. 
%Intensity image is an image the value of each pixel carries only grayscale information. 
%When lighting sources changes strongly, intensity image cause some noise and redundant information. 
%It can cause negative effect when mapping between image and depth information.  
%/XXX

%說明貢獻
To overcome the problems caused by different lighting sources, 
%The color information is much less sensitive to environment lighting variation than the grayscale information.
%本篇論文的大綱 
we propose the use of multiple factors in the tensor model to reduce the effect from uncontrolled environmental lights in this paper.
Our method considers color channels in a single color image, which are less affected by different lighting sources. The faces scanned by an off-the-shelf 
    depth camera such as Microsoft Kinect are used to construct the initial database. 
During the training phase, two tensor models are constructed for the color image and depth information respectively. 
Canonical correlation analysis (CCA) is used to establish the mapping between colors and depths. 
To reconstruct a face from a color image, the color tensor is first constructed, then depths are estimated using the mapping computed by CCA. 

The rest of this paper is organized as follows. 
Section \ref{Related} briefly introduces related work. 
Section \ref{app} outlines the proposed approach. 
The experiment results are discussed in section \ref{exp}. 
We conclude this paper in Section \ref{sum}.