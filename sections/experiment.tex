%\section{Experiment}\label{exp}


%講解實驗規劃與如何呈現數據
	
	根據這些我們使用定位方法的描述,我們想要改善室內定位的覆蓋率以及定位的精準度。為了驗證實驗所改善的數據,分成兩個實
驗環境來說明方法所改善的數據:(1)可以控制的實驗環境、(2)一般室內定位的環境。

	首先製造一個可以控制特徵點數量的環境,在這個環境中我們驗證每個固定距離內根據 SIFT 所涵蓋的特徵點數量作比較,增加
與物體的固定距離算出每個距離中的平均定位誤差,再算出定位誤差範圍的覆蓋率與傳統的照片影像定位作比較。在可以控制的定位環
境下我們根據這些實驗方法說明改善的成果,再把方法放建築一般實際的室內環境中作比較,最後呈現出改善的平均定位誤差與增加環
境所能定位的覆蓋範圍。

%1. Controlled Enviornment 



%2. Indoor Localization
