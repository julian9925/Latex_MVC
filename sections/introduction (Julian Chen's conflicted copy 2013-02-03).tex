%%\section{Introduction}
%%人臉重建會遇到的問題及方法種類
%3D face reconstruction from 2D images is a challenging problem in computer vision.
%%, because of inaccuracy, time-varying, time-consuming, unknown surface information. 
%Over the years, several 3D face reconstructing approaches from 2D images have been proposed. Methods used to acquire depth information of faces including stereo vision\cite{seitz2006comparison}, structure from motion \cite{tomasi1992shape}, 3D morphable model based methods \cite{romdhani2003efficient}, shape from shading techniques \cite{zhang1999shape}, and tensor \cite{vasilescu2002multilinear, kolda2009tensor}.
%
%
%% stereo vision 方法特性
%In methods using stereo vision, selecting proper corresponding points is crucial because inaccurate corresponding points greatly affect the accuracy of depth estimation.  Corresponding points can be used to measure \textit{disparity}, which is the difference between the corresponding points of left and right images. Disparity is then used to evaluate depth, which is the distance to the target object.
%%Okutomi														
%Okutomi \cite{okutomi1993multiple} proposed a method to match corresponding points by computing the sum of squared-difference (SSD) with different baselines by a lateral displacement of a camera.
%%Hansen
%Recently, Hansen \cite{hansen20103d} developed a new 3D face capture algorithm using Photometric Stereo (PS) device with visible light or near infrared light (NIR). In general, the NIR light sources offer better performance than the visible light.
%
%%structure from motion 方法特性
%In methods using structure from motion, the three-dimensional structure of an object is found by analyzing local motion signals over time. It can recover 3D structure from the projected 2D motion field of a moving object.
%%Torresani
%Torresani et al.\cite{torresani2008nonrigid} proposed a reconstruction method using a Probabilistic Principal Components Analysis (PPCA) shape model and an estimation algorithm that estimates 3D shape and motion for each time moment simultaneously. 
%%3D morphable model 
%In 3D morphable model (3DMM) based methods, by given a large database of 3D face models, any arbitrary face can be generated by morphing in the database.		
%%Romdhani
%Romdhani \cite{romdhani2003efficient} proposed a 3D alignment algorithm to recover the facial shape and texture parameters from the appearance features. However, the 3D face alignment requires manual initialization and the reconstructing speed was not yet suitable for real time face recognition systems. 					
%%Hu
%To speed up the fitting algorithm, Hu \cite{hu2004automatic} proposed a fully automatic linear algorithm to recover the shape information according to sparsely corresponded 2D facial feature points, but the texture geometry extracted from a single image could not cover the whole face.
%%Le 
%Le et al.\cite{le2010accurate} proposed an algorithm which reconstructs the 3D shape and texture of human faces from stereo images captured from calibrated cameras. Using PCA-based morphable model and iterative optimization process, their algorithm fits arbitrary views of faces to a dense 3D model from a sparse 3D point set.						
%%shape from shading
%Shape from shading (SFS) methods use shading information to reconstruct the three-dimensional object. It needs to solve unknown lighting source and albedo (reflection coefficient) from object surface.
%%Kemelmacher
%Kemelmacher \cite{kemelmacher20113d} proposed a novel method for recovering 3D faces from a single image and single 3D reference model of a different person’s face.
%
%%介紹tensor
%In recent years, several tensor-based methods are proposed for face recognition and reconstruction \cite{vasilescu2002multilinear, kolda2009tensor}. 
%%介紹Lei的tensor model及使用NIR image的優點
%Lei et al. \cite{lei2008face} recovered faces from a single image. \cite{lei2008face} uses a single near infrared (NIR) image as input and constructs a mapping from the NIR tensor space to 3D tensor space.
%
%%說明動機
%In the NIR imaging system environment, \cite{lei2008face} is able to obtain good performance. 
%However, because of the fact that cost constrain and the NIR imaging system is not easy to obtain, the grayscale images are substituted for NIR images. 
%When we use the grayscale images as input, different lighting sources and other factors due to cameras affect the computed result. 
%The grayscale image is an image the value of each pixel carries only intensity information. 
%When lighting sources changes strongly, grayscale images cause some noisy and redundant information, because of grayscale images are sensitive to environmental lighting variation. 
%It can cause negative effect when mapping between image and depth information.
%
%%XXX Although \cite{lei2008face} is able to obtain good performance in the NIR imaging system environment, when %intensity images are substituted for 
%%    NIR image, lighting sources and other factors due to cameras degrade the computed result. 
%%In uncontrolled environment, a person's intensity image may appear differently because of intensity image are %sensitive to environment lighting variation. 
%%Intensity image is an image the value of each pixel carries only grayscale information. 
%%When lighting sources changes strongly, intensity image cause some noise and redundant information. 
%%It can cause negative effect when mapping between image and depth information.  
%%/XXX
%
%%說明貢獻
%To overcome the problems caused by different lighting sources, 
%%The color information is much less sensitive to environment lighting variation than the grayscale information.
%%本篇論文的大綱 
%we propose the use of multiple factors in the tensor model to reduce the effect from uncontrolled environmental lights in this paper.
%Our method considers color channels in a single color image, which are less affected by different lighting sources. The faces scanned by an off-the-shelf 
%    depth camera such as Microsoft Kinect are used to construct the initial database. 
%During the training phase, two tensor models are constructed for the color image and depth information respectively. 
%Canonical correlation analysis (CCA) is used to establish the mapping between colors and depths. 
%To reconstruct a face from a color image, the color tensor is first constructed, then depths are estimated using the mapping computed by CCA. 
%
%The rest of this paper is organized as follows. 
%Section \ref{Related} briefly introduces related work. 
%Section \ref{app} outlines the proposed approach. 
%The experiment results are discussed in section \ref{exp}. 
%We conclude this paper in Section \ref{sum}.


	室內定位研究大部分是利用訊號接收作三角定位,但是需要設備上的支援,如:紅外線、超聲波、Wifi 等設備,訊號的不穩定也會影響定位的誤差。
之後有研究把影像特徵點當作訊號作定位研究,影像定位可以減少訊號不穩定導致的定位誤差,但是照片取樣分布不均及相機角度上的限制,使得室內定位
可定位的範圍受到很大的限制。我們方法利用照片組建出3D點雲環境,在3D環境內拍攝虛擬相片作定位,彌補訊號不穩定的情形及增加可定位的範圍。在相
關研究的章節中我們分成三個階段來說明本篇方法中使用的研究技術。首先基於現今室內定位的研究作探討,再來說明本文方法中3D環境技術的相關研究,
最後說明如何在照片中找到特徵點及特徵點匹配的方法。

\section{室內定位相關研究}

	現在室內定位的技術多用於訊號接收定位,需要設備上的支援,下面介紹幾種常用的技術:紅外線技術:紅外線室內定位技術定位的原理是利用IR標識的
紅外射線,通過安裝在室內的光學傳感器接收進行定位。雖然紅外線具有相對較高的室內定位的精準度,但是由於光線不能穿過障礙物,當標識放在口袋裡或者
有牆壁及其他遮擋時就不能正常工作。因此,紅外線只適合短距離傳播,而且容易被房間內的燈光所干擾,在定位上有一定的限制。

  超聲波技術:超聲波測距主要採用反射式測距法,通過三角定位等算法確定物體的位置,即發射超聲波並接收由被測物產生的回波,根據回波與發射波的時
間差計算出待測距離。以\cite{Bekkerman2006}\cite{TenaRuiz2004}\cite{Gro�mann20011}的方法來說,當同時有3個或3個以上不在同一直線上的應
答器做出回應時,可以根據相關計算確定出被測物體所在的二維坐標系下的位置。超聲波定位整體定位精度較高,結構簡單,但超聲波需要大量的底層硬體設施
投資,成本太高。

  藍牙技術:藍牙技術通過測量訊號強度進行定位。這是一種短距離低功耗的無線傳輸技術,在室內安裝適當的藍牙裝置,配置成多用戶的連接模式,並保證
此藍牙設備始是這個區域內的主要設備,就可以獲得用戶的位置訊息。藍牙技術主要應用於小範圍定位,例如大廳或倉庫。藍牙室內定位技術最大的優點是設備
體積小、易於作在PDA、PC以及手機中,因此很容易推廣普及。但不足在於藍牙器材和設備的價格比較昂貴,而且對於復雜的空間環境,藍牙系統的穩定性稍差
,受信號干擾大。 

  Wi-Fi技術:無線區域網路(WLAN)是一種訊息平台,可以在廣泛的應用領域內實現複雜的大範圍定位、監測和追踪任務,而網路節點自身定位是大多
數應用的基礎和前提。以\cite{Chin-HengLim2007}\cite{JahyoungKo2011}\cite{KeWang2011}為例,當前比較流行的Wi-Fi定位是IEEE
802.11無線網路標準的定位解決方案。易於安裝,只需要很少的Access Point(AP)設備,也能採用相同的底層無線網絡架構,定位精準度高。但是,如
果定位的測算僅僅依賴於哪個AP點最近,而不是依賴合成的訊號強度,那麼在樓層定位上很容易出錯。目前,它應用於小範圍的室內定位,成本較低。但無論
是用於室內還是室外定位,容易受到其他信號的干擾,從而影響其精度,定位器的能耗也較高。
	
	近年來由於RFID技術的興起,也把定位運用在RFID上。RFID也是以訊號為定位的依據,好處是不需要像Wi-Fi需要AP設備支援,無須受到網路上的限
制。在\cite{Zhou2009}提到,RFID定位方法與Wi-Fi相近,但是設備更為簡單且運用相當廣泛,像是運用在道路安全偵測、醫療保健或是生產線的流程監
控都可運用。但RFID的訊號比起Wi-Fi更容易被受到干擾,且訊號傳送距離也比Wi-Fi來的短,無法做到很精確的定位。

  ZigBee技術:ZigBee是一種新興的短距離、低速率無線網絡技術,可以用於室內定位。它有自己的無線電標準,在數千個微小的感測器之間互相傳送訊息
以實現定位。這些感測器只需要很少的能量,以接力的方式通過無線電波將數據從一個感測器傳到另一個感測器,所以它們的傳送效率非常高。ZigBee最顯著的
特點是它的低功耗和低成本。

  除了以上提及的定位技術,還有基於電腦視覺、影像分析、磁場定位等。影像分析定位就是我們這次所研究的目標,目的可以減低設備上的使用成本,也包含
蓋更多關於整體環境的資訊。