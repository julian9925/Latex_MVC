\section{研究背景介紹}
%影像特徵描述與定位應用
	機器人相關領域研究在近年來研究相關廣泛,有許多的研究應用在生活上面,像是定位、導航、視覺感測等,在定位的研究上像是現在的GPS定位,在只要
有衛星訊號的環境上就可以運用定位系統,在陌生的環境下可以藉由定位了解到自身所在位置相關的資訊。在有定位資訊的輔助下,藉由路徑規劃,即可完成運送
、傳遞的任務,像是車用導航,輔助駕駛到達目的地。室外定位除了利用GPS外,也有利用紅外線及雷達等系統,但GPS最為廣泛,因為擁有全世界的視野,不須
其他儀器輔助,其他像是紅外線及雷達等的儀器,遍布沒有這麼廣泛,較無法運用在日常生活中,較常用在私人用途,像是飛機、軍艦等的導航系統。

	室內定位的技術多用於訊號接收定位,需要設備上的支援,像是紅外線、聲納\cite{Bekkerman2006}、藍牙、Wi-Fi\cite{Chin-HengLim2007}
以及RFID\cite{Zhou2009}等室內定位的技術,這些技術的共同缺點都是訊號容易受到環境的干擾,使得無法精確地定出定位位置,誤差範圍過大(5~10m),
為了替代訊號在定位上的使用,之後室內定位的研究轉往在影像上的特徵當作定位來源。影像定位藉由影像上的特徵作為定位處理的依據,影像特徵得擷取分為兩
大部分,一為區域導向(Region based),一為特徵點導向(feature based)\cite{Zitová03}。區域導向代表在兩張相對圖片中,找出相對應畫素的關係性,
像是以梯度向量(Gradient)強度,或是色彩像素上的強度作為特徵描述。另外一種特徵點導向是以lowe\cite{Lowe2004}所提出尺度不變特徵轉換(SIFT)為前
提,所使用擷取特徵的方法。

	SIFT特徵點須具備以下幾個條件:特徵點的識別度高,具有獨特性、容易匹配、對各種的破壞具有不變性。基於這上述三項理由,使得SIFT比一般區域導向的特徵
描述常用在定位演算法中,一個完整的特徵點演算法可以分為兩個部份,一為特徵點擷取,一為特徵點描述。在一個特徵點匹配的演算法中
	
	
%	原理是利用IR標識的
%紅外射線,通過安裝在室內的光學傳感器接收進行定位。雖然紅外線具有相對較高的室內定位的精準度,但是由於光線不能穿過障礙物,當標識放在口袋裡或者
%有牆壁及其他遮擋時就不能正常工作。因此,紅外線只適合短距離傳播,而且容易被房間內的燈光所干擾,在定位上有一定的限制。

%  超聲波技術:超聲波測距主要採用反射式測距法,通過三角定位等算法確定物體的位置,即發射超聲波並接收由被測物產生的回波,根據回波與發射波的時
%間差計算出待測距離。以\cite{Bekkerman2006}\cite{TenaRuiz2004}\cite{Gro�mann20011}的方法來說,當同時有3個或3個以上不在同一直線上的應
%答器做出回應時,可以根據相關計算確定出被測物體所在的二維坐標系下的位置。超聲波定位整體定位精度較高,結構簡單,但超聲波需要大量的底層硬體設施
%投資,成本太高。
%
%  藍牙技術:藍牙技術通過測量訊號強度進行定位。這是一種短距離低功耗的無線傳輸技術,在室內安裝適當的藍牙裝置,配置成多用戶的連接模式,並保證
%此藍牙設備是這個區域內的主要設備,就可以獲得用戶的位置訊息。藍牙技術主要應用於小範圍定位,例如大廳或倉庫。最大的優點是設備
%體積小、易於作在PDA、PC以及手機中,因此很容易推廣普及。但不足在於對於復雜的空間環境,藍牙系統的穩定性稍差,受信號干擾大。 
%
%  Wi-Fi技術:無線區域網路(WLAN)是一種訊息平台,可以在廣泛的應用領域內實現複雜的大範圍定位、監測和追踪任務,而網路節點自身定位是大多
%數應用的基礎和前提。以\cite{Chin-HengLim2007}\cite{JahyoungKo2011}\cite{KeWang2011}為例,當前比較流行的Wi-Fi定位是IEEE
%802.11無線網路標準的定位解決方案。易於安裝,只需要很少的Access Point(AP)設備,也能採用相同的底層無線網絡架構,定位精準度高。但是,如
%果定位的測算僅僅依賴於哪個AP點最近,而不是依賴合成的訊號強度,那麼在樓層定位上很容易出錯。目前,它應用於小範圍的室內定位,成本較低。但無論
%是用於室內還是室外定位,容易受到其他信號的干擾,從而影響其精度,定位器的能耗也較高。
%	
%	近年來由於RFID技術的興起,也把定位運用在RFID上。RFID也是以訊號為定位的依據,好處是不需要像Wi-Fi需要AP設備支援,無須受到網路上的限
%制。在\cite{Zhou2009}提到,RFID定位方法與Wi-Fi相近,但是設備更為簡單且運用相當廣泛,像是運用在道路安全偵測、醫療保健或是生產線的流程監
%控都可運用。但RFID的訊號比起Wi-Fi更容易被受到干擾,且訊號傳送距離也比Wi-Fi來的短,無法做到很精確的定位。
%
\section{研究動機}

	室內定位訊號受到環境上的阻礙,在比較複雜的室內環境下,會因為訊號不穩定而導致定位精準度下降,又受限於設備上的限制,無法落實在日常生活上。
我們想避免運用到需要訊號定位的設備,利用日常可隨身攜帶的相機當作我們定位的資訊,藉由影像特徵的幫助下完成室內定位。影像定位雖有比訊號定位較高的
精準度,但是受限於拍照位置與角度上的限制,不能完整的涵蓋室內環境的資訊,在我們的研究想還原定位的3D環境,利用近年來常用的Kinect深度感測攝影機
幫助我們建置3D環境,如何利用3D環境拍攝照片作室內定位,為我們研究中想要解決的問題。
	
\section{研究目的}
	
	為使得影像定位能有更準確且更高覆蓋率的的定位,我們結合了3D環境的技術幫助我們達成目的,在重建的定位環境下設置虛擬相機取得照片,藉由均勻分布
在定位環境下,取得比相機拍攝更好的影像來源,除了均勻的分布位置外,我們根據虛擬相機所在的位置作深度檢測,當距離障礙物過近的情況下,能夠自動的旋轉
相機角度拍攝更好的虛擬影像,希望藉由虛擬影像定位提升影像定位的覆蓋率與精準度。在最後定位實驗的數據分析,利用虛擬影像定位與傳統影像定位數據作比較
,證明我們的方法在室內定位中有更好的效果,希望能夠在影像定位的研究中,提供新的解決方法與新的研究發展方向。

\section{論文架構}